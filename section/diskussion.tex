\chapter{Diskussion og vurdering}\label{kap:diskussion}
Som helhed virkede de fleste af tilgangene for at løfte opgaven tilfredsstillende.
Arbejdsmetoder og workflow viste sig at fungere som forventet.
Planlægning med stramme deadline blev overholdt, og gav anledning til overblik i projektet.

I følgende afsnit vil en mere dybdegående diskussion og vurdering gennemgås med relation til de enkelte afsnit. 

\subsection{Fysisk model}
Ved at anvende klassisk fritlegeme analyse af pendulet lykkedes det at få en reguleringsmæssig brugbar beskrivelse af fysikken. 
Det var dog en lang og besværlig vej at løse problemet på, da man med anden viden, fx Lagrange og statespace analyse ville frembringe et brugbart resultat.
Der blev anvendt en del tilnærmelser i udledningen. 
Da alle tilnærmelser er velargumenteret, vurderes det at deres anvendelse er valide og det endelige resultat anses for korrekt.   

\subsection{Vognens overførelsesfunktioner}
Da hele vognen med pendul og motor var arvet fra en tidligere projekt gruppe, var en af udfordringerne af finde frem til den dynamik der beskriver disse dele.
Den anvendt metode viste sig at fungere og det endelige resultat kunne, ud fra en velovervejet simplificering, anvendes i den samlede systembeskrivelse der ligger til grund for bestemmelsen af regulatoren.
Til fremtidige forbedringer ville udskiftning af fx motor kunne føre til en mere dybdegående beskrivelse af systemets overførelsesfunktion, og derved give en endnu bedre regulering og stabilitet.

\subsection{Sensor}
Ud fra analysering af sammenkopling mellem to solenoider, lykkedes det vha. forskellige teorier - her i blandt Faradays lov, at lave en velfungerende model af et sensorsystem. 
Det mest krævende, var at lave et præcist resultat for magnetfeltet, da dette krævede flere teknikker, heriblandt mange matematiske udtryk.
Faktorerne for den inducerede afsenderspole, var meget diskuteret, da strømmen fra frekvensgeneratoren afhang meget af styrken på magnetfeltet denne dannede.
Herudover også trådtykkelse og længde. 
Da alle teoretiske beregninger blev eftervist i laboratoriet, antages der med grov tilnærmelse, at de endelige resultater antages for korrekte.

Til dimensionering af timer-kredsen blev der anvendt vedlagte ligninger i databladet.
Det var herfra muligt at dimensionerer komponentstørrelser der ville lave et signal med en frekvens der lå tæt på den ønskede.
Desværre da der blev lavet mållinger på kredsløbet viste det sig at frekvens varierede noget mere end fra da det blev testet på breadboard.
Dette blev først opdaget hen mod slutning af projektet, så der var ikke tid til at fejl finde.
Fejlen skyldes højst sandsynligt et forkert komponent.
Endvidere viste det sig at udgangssignalet ikke oscillerede mellem en konstant positiv og negativ spænding af samme størrelse.
Årsagen til dette er endnu ikke bekendt.
En mulig forbedring kunne være at designe en buckconverter da denne kan opnå den duty-cycle tættere på 50\%, hvilket medvirker at magnetfeltet varierer lige hurtigt i begge retninger.

\subsection{Signalbehandling}
Ved løbende test af hvert kredsløb på breadboard, blev de endelige kredsløb dimensioneret. På trods af støj i breadboards, virkede kredsløbene efter hensigten. For at undgå støj i de endelige kredsløb, blev der påsat afkoblingskondensatorer alle nødvendige steder.

En mulig forbedring af signalbehandlingen blev undersøgt, hvor der blandt andet blev kigget på en aktiv ensretter kreds. Problemet med den var dog der skulle bruges to operationsforstærkere til én ensretter. Det endelige kredsløb indeholder to aktiv båndpasfiltre, hvilket kunne være undgået, hvis de i stedet var passive. Problemet var, at det med det passive ikke er muligt at forstærke signalet.

Grundet approksimering af komponentværdier, kom der en afvigelse fra de teoretiske forventninger.



\subsection{Motorstyring}
Det viste sig tilstrækkeligt at anvende bi-directional DC-motor driver typologien som styring af motoren. 
Den negative feedback sørgede for en velfungerende regulering af strømmen igennem motoren, som var vigtig for at reguleringssystemet kunne fungere og give den ønskede stabilitet.
Kompleksiteten omkring motoren var stor, så derfor var mange tilnærmelser nødvendige.
Disse tilnærmelser anses dog for at være velbegrundet, men deres omfang var dækkende vides dog ikke med sikkerhed.
Under projekt arbejdet, viste en manglede strøm igennem motoren at være et problem mht. response hastigheden af vognen overfor udefrakommende forstyrrelser.     
En ændring af design ved indførelse af en ekstra $\pm 9 \si{\volt}dc$ forsyning til motorstyringens op-amp løste det problem.
Det ville på sigt, give en bedre stabilitet at udskift den nuværende motorstyring, men en der kan håndtere mere effekt, men siden den nuværende motor effekt giver anledning til mistet vejgreb fra hjul til underlag, vil en opdatering af hul være mere vigtigere. 


\subsection{Regulering}




