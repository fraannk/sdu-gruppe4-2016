\chapter{Diskussion og vurdering}\label{kap:diskussion}

\subsection{Fysisk model}
Ved at anvende klassisk fritlegeme analyse af pendulet lykkedes det at få en reguleringsmæssig brugbar beskrivelse af fysikken. 
Det var dog en lang og besværlig vej at løse problemet på, da man med anden viden, fx Lagrange og statespace analyse ville frembringe et brugbart resultat.
Der blev anvendt en del tilnærmelser i udledningen. 
Da alle tilnærmelser er velargumenteret, vurderes det at deres anvendelse er valide og det endelige resultat anses for korrekt.   

\subsection{Vognens overførelsesfunktioner}

\subsection{Sensor}
Ud fra analysering af sammenkopling mellem to solenoider, lykkedes det vha. forskellige teorier - her i blandt Faradays lov, at lave en velfungerende model af et sensorsystem. 
Det mest krævende, var at lave et præcist resultat for magnetfeltet, da dette krævede flere teknikker, heriblandt mange matematiske udtryk.
Faktorerne for den inducerede afsenderspole, var meget diskuteret, da strømmen fra frekvensgeneratoren afhang meget af styrken på magnetfeltet denne dannede.
Herudover også trådtykkelse og længde. 
Da alle teoretiske beregninger blev eftervist i laboratoriet, antages der med grov tilnærmelse, at de endelige resultater antages korrekte.




\subsection{Signalbehandling}

\subsection{Motorstyring}

\subsection{Regulering}

\subsection{Helhedsvurdering}

