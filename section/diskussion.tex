\chapter{Diskussion og vurdering}\label{kap:diskussion}

\subsection{Fysisk model}
Ved at anvende klassisk fritlegeme analyse af pendulet lykkedes det at få en reguleringsmæssig brugbar beskrivelse af fysikken. 
Det var dog en lang og besværlig vej at løse problemet på, da man med anden viden, fx Lagrange og statespace analyse ville frembringe et brugbart resultat.
Der blev anvendt en del tilnærmelser i udledningen. 
Da alle tilnærmelser er velargumenteret, vurderes det at deres anvendelse er valide og det endelige resultat anses for korrekt.   

\subsection{Vognens overførelsesfunktioner}

\subsection{Sensor}
Ud fra analysering af sammenkopling mellem to solenoider, lykkedes det vha. forskellige teorier - her i blandt Faradays lov, at lave en velfungerende model af et sensorsystem. 
Det mest krævende, var at lave et præcist resultat for magnetfeltet, da dette krævede flere teknikker, heriblandt mange matematiske udtryk.
Faktorerne for den inducerede afsenderspole, var meget diskuteret, da strømmen fra frekvensgeneratoren afhang meget af styrken på magnetfeltet denne dannede.
Herudover også trådtykkelse og længde. 
Da alle teoretiske beregninger blev eftervist i laboratoriet, antages der med grov tilnærmelse, at de endelige resultater antages korrekte.


Til dimensionering af timer-kredsen blev der anvendt vedlagte ligninger i databladet.
Det var herfra muligt at dimensionerer komponentstørrelser der ville lave et signal med en frekvens der lå tæt på den ønskede.
Desværre da der blev lavet mållinger på kredsløbet viste det sig at frekvens varierede noget mere end fra da det blev testet på breadboard.
Dette blev først opdaget hen mod slutning af projektet, så der var ikke tid til at fejl finde.
Fejlen skyldes højst sandsynligt et forkert komponent.
Endvidere viste det sig at udgangssignalet ikke oscillerede mellem en konstant positiv og negativ spænding af samme størrelse.
Årsagen til dette er endnu ikke bekendt.
En mulig forbedring kunne være at designe en buckconverter da denne kan opnå den duty-cycle tættere på 50\%, hvilket medvirker at magnetfeltet varierer lige hurtigt i begge retninger.

\subsection{Signalbehandling}
Ved løbende test af hvert kredsløb på breadboard, blev de endelige kredsløb dimensioneret. På trods af støj i breadboards, virkede kredsløbene efter hensigten. For at undgå støj i de endelige kredsløb, blev der påsat afkoblingskondensatorer alle nødvendige steder.

En mulig forbedring af signalbehandlingen blev undersøgt, hvor der blandt andet blev kigget på en aktiv ensretter kreds. Problemet med den var dog der skulle bruges to operationsforstærkere til én ensretter. Det endelige kredsløb indeholder to aktiv båndpasfiltre, hvilket kunne være undgået, hvis de i stedet var passive. Problemet var, at det med det passive ikke er muligt at forstærke signalet.

Grundet approksimering af komponentværdier, kom der en afvigelse fra de teoretiske forventninger.

\subsection{Motorstyring}

\subsection{Regulering}

\subsection{Helhedsvurdering}

