\chapter{Diskussion og vurdering}\label{kap:diskussion}

\subsection{Fysisk model}
Ved at anvende klassisk fritlegeme analyse af pendulet lykkedes det at få en reguleringsmæssig brugbar beskrivelse af fysikken. 
Det var dog en lang og besværlig vej at løse problemet på, da man med anden viden, fx Lagrange og statespace analyse ville frembringe et brugbart resultat.
Der blev anvendt en del tilnærmelser i udledningen. 
Da alle tilnærmelser er velargumenteret, vurderes det at deres anvendelse er valide og det endelige resultat anses for korrekt.   

\subsection{Vognens overførelsesfunktioner}
Da hele vognen med pendul og motor var arvet fra en tidligere projekt gruppe, var en af udfordringerne af finde frem til den dynamik der beskriver disse dele.
Den anvendt metode viste sig at fungere og det endelige resultat kunne, ud fra en velovervejet simplificering, anvendes i den samlede systembeskrivelse der ligger til grund for bestemmelsen af regulatoren.
Til fremtidige forbedringer ville udskiftning af fx motor kunne føre til en mere dybdegående beskrivelse af systemets overførelsesfunktion, og derved give en endnu bedre regulering og stabilitet.

\subsection{Sensor}


\subsection{Signalbehandling}

\subsection{Motorstyring}


\subsection{Regulering}



\subsection{Helhedsvurdering}

