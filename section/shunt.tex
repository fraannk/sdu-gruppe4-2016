\section{Shunt regulator}\label{sec:shunt}



\begin{itemize}
	\item Konstant x antal volt til frekvensgeneratoren
	\item Tager højde for voltage sag i LiPo batterier
\end{itemize}


\subsection{Design}
En shunt regulator er en spændingsregulator der benytter en zenerdiode til at regulere spændings niveauet.
Shunt regulatoren har den fordel at uanset ripplen på indgangsspædningen så vil den forsøge at holde udgangsspændingen med en så lille ripple som muligt.
Der skal her designes en shunt regulator der får $8.4 \si{\volt}$ på indgangen, og laver $7 \si{\volt}$ på udgangen.
Spændingen på udgangen skal være henholdsvis $\pm 7 \si{\volt}$ da dette skal bruges til at forsyne operationsforstærkerne i de næste kredsløb.

\subsection{Beregninger}


\begin{align}
	V_{in} & = 8.4 \si{\volt} \nonumber \\
	V_{out} & = 7 \si{\volt} \nonumber \\
	I_{in} & = 20\si{\milli\ampere} \nonumber \\
	I_Z & = 5 \si{\milli\ampere} \nonumber \\
	I_{out} & = I_{in} - I_Z \\
	I_{out} & = 15 \si{\milli\ampere} \nonumber \\
	R_1 & = \frac{V_{in} - V_{out}}{I_Z + I_{out}} \label{eq:RegulatorModstand} \\
	R_1 & = 70 \si{\ohm} \nonumber \label{eq:RegulatorModstandBeregnet}
\end{align}

R1 er lig R2
Den første R1 er den beregnede
\begin{align}
	R_1 & = 68 \si{\ohm} \nonumber \\
	V_{out} & = V_{in} - I_{out} \cdot R_s - I_z \cdot R_s \\
	V_{out} & = 7.04 \si{\volt} \label{eq:RegulatorBeregnetPotentiale} 
\end{align}