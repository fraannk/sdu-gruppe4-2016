\section{Shunt regulator}\label{sec:shunt}
En shunt regulator er en spændingsregulator der benytter en zenerdiode til at regulere spændings niveauet.
Idet forsyningsbatterierne ikke kan holde en konstant udgangsspænding, kan operationsforstærkerne ved høj forstærkning gå i mætning. 
For at undgå det, anvendes shunt regulatoren, da den kan holde en nogenlunde konstant spænding, på trods af batteriernes faldende udgangsspænding.

\subsection{Design}
Der skal her designes en shunt regulator der får $8.4 \si{\volt}$ på indgangen, og laver $7\si{\volt}$ på udgangen.
Der bruges to shunt regulatore, idet der forsyningerne til operationsforstærkerne kræver en positiv og en negativ forsyning på 7 \si{\volt}.
\husk{Kenneth}{Billede af shunt regulator kreds}

\subsection{Beregninger}
Indgangsspændingen fra batterierne er $V_{in} = \pm 8.4 \si{\volt}$. 
For at overskueligegøre beregningerne, anvendes kun den positive spændingen, hvoraf komponentværdierne antager samme værdi for den negative spænding, dog bliver zenerdioden forspændt i modsat retning.
Udgangsspændingen ønskes at $V_{out} = 7 \si{\volt}$. 
Indgangsstrømmen vælges til $I_{in} = 20\si{\milli\ampere}$, og bruges til at bestemme udgangsstrømmens interval.
Zenerdiodens strøm og modstand findes i databladet $I_Z = 5 \si{\milli\ampere}$ $r_z = 15 \si{\ohm}$.

\husk{Kenneth}{Datablad på zenerdiode. (der henvises til den)}
\husk{Kenneth}{Forkert zenerdiode valgt i stykliste?}
For at der skal ligge en konstant spænding på $7 \si{\volt}$ udregnes først spændingsfaldet over dioden ved at bruge formel 3.21 \cite[Side. 146]{Sedra19uu} formelen for spændingen over dioden er den samme som udgangsspændingen på loaden. 
\begin{align}
	V_o & = V_{Z0} + r_z I_z \\
	V_{Z0} & = 6.925 \si{\volt}
	\end{align}
For at beregne modstanden på udgangen af kredsen anvendes ligning 3.27 \cite[Side. 149]{Sedra19uu}.

\begin{align}
	R_1 & = \frac{V_{in}-V_{Z0}-r_z I_z}{I_z+I_{out}} \label{eq:RegulatorModstand} \\
	R_1 & = 70 \si{\ohm} \nonumber \label{eq:RegulatorModstandBeregnet}
\end{align}

Det kan ikke lade sig gøre at få en modstand på $70 \si{\ohm}$, derfor vælges $R_1 = 68 \si{\ohm}$.
Heraf udregnes udgangsspændingen for den nye modstand, for at se hvordan dette påvirker systemet.
Til dette anvendes ligning 3.24 \cite[Side. 149]{Sedra19uu}.
\begin{align}
	V_{out} & = V_{Z0} \cdot \left( \frac{R_1}{R_1+r_z} \right) + V_{in} \cdot \left( \frac{r_z}{R_1+r_z} \right) - I_L \cdot \left( \frac{r_z \cdot R_1}{r_z+R_1} \right) \\
	V_{out} & = 7.01 \si{\volt} \label{eq:RegulatorBeregnetPotentiale} 
\end{align}

Ovenstående beregninger er udført for en BZX79C6V8 diode. 
Da der blev lavet en endelig komponentliste blev der ved et uheld valgt en BZX79C6V2 diode i stedet for. 
Forskellen her er at den nye diode maksimalt kan holde en konstant spænding på $6.6 \si{\volt}$ i udgangsspænding og diode modstanden er $r_z = 10 \si{\ohm}$.
Ved tilsvarende udregninger findes det frem til at udgangsspændingen er $6.66 \si{\volt}$.