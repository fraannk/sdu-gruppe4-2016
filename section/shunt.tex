\section{Shunt regulator}\label{sec:shunt}
En shunt regulator er en spændingsregulator der benytter en zenerdiode til at regulere spændings niveauet.
Idet forsyningsbatterierne ikke kan holde en konstant udgangsspænding, kan operationsforstærkerne ved høj forstærkning gå i mætning. 
For at undgå det, anvendes shunt regulatoren, da den kan holde en nogenlunde konstant spænding, på trods af batteriernes faldende udgangsspænding.

\subsection{Design}
Der skal her designes en shunt regulator der får $8.4 \si{\volt}$ på indgangen, og laver $7\si{\volt}$ på udgangen.
Der bruges to shunt regulatore, idet der forsyningerne til operationsforstærkerne kræver en positiv og en negativ forsyning på 7 \si{\volt}.
\husk{Kenneth}{Billede af shunt regulator kreds}

\subsection{Beregninger}
Indgangsspændingen fra batterierne er $V_{in} = \pm 8.4 \si{\volt}$. 
For at overskueligegøre beregningerne, anvendes kun den positive spændingen, hvoraf komponentværdierne antager samme værdi for den negative spænding, dog bliver zenerdioden forspændt i modsat retning.
Udgangsspændingen ønskes at $V_{out} = 7 \si{\volt}$. 
Indgangsstrømmen vælges til $I_{in} = 20\si{\milli\ampere}$, og bruges til at bestemme udgangsstrømmens interval.
Zenerdiodens strøm findes i databladet $I_Z = 5 \si{\milli\ampere}$.

\husk{Kenneth}{Datablad på zenerdiode. (der henvises til den)}
Da der skal være en konstant strøm i zenerdioden, for at opretholde den rette spænding, udregnes ligning \ref{eq:Udgangsstroem},
\begin{align}
	I_{out} & = I_{in} - I_Z \label{eq:Udgangsstroem}
\end{align}
og der fås en udgangsstrøm på $I_{out} = 15 \si{\milli\ampere}$.
Modstanden udregnes med ligning \ref{eq:RegulatorModstand}, 
\begin{align}
	R_1 = R_2 & = \frac{V_{in} - V_{out}}{I_Z + I_{out}} \label{eq:RegulatorModstand}
\end{align}
Modstanden udregnes til $70 \si{\ohm}$, hvoraf den tilnærmede værdi bliver $68 \si{\ohm}$.
\begin{align}
	V_{out} & = V_{in} - I_{out} \cdot R_1 - I_z \cdot R_1
\end{align}
Udgangsspændningen bliver da $V_{out} = 7.04 \si{\volt}$.

Ovenstående beregninger er udført for en BZX79C6V8 diode. 
Da der blev lavet en endelig komponentliste blev der ved et uheld valgt en BZX79C6V2 diode i stedet for. 
Forskellen her er at den nye diode maksimalt kan holde en konstant spænding på $6.6 \si{\volt}$ i udgangsspænding.
Ved tilsvarende udregninger findes det frem til at udgangsspændingen er $6.5 \si{\volt}$.