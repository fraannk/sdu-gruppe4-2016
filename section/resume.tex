I dette projekt prøves, ved hjælp af klassisk reguleringsteknik, fritlegeme analyse af fysiske systemer, et elektromagnetisk sensorsystem samt analogt signalbehandling, at stabilisere et inverteret pendul monteret på en vogn der drives af en DC-motor.
Systemet forsynes med LiPo batterier for øget bevægelsesfrihed. 
Stabiliteten opnås ved en P-Lead kompensator der dimensioneres på baggrund af en dybdegående analyse af systemets enkelte dynamikker. 
Regulatoren forsynes med udgangssignalet fra den analoge signalbehandling bestående af filtre, ensretning og signalsammenligning. 
Ved hjælp af en bi-directional motorstyring, korrigeres pendulets position. 
Positionen registeres ved hjælp af veldimensionerede spoler og induktiv kobling i det elektromagnetiske sensorsystem.    
I projektet gennemgås og dokumenteres løsningsmodeller, teorier opstilles og efterprøves samt testresultater analyseres.
Arbejdet munder ud i en funktionel vogn, der indenfor de fremsatte krav kan holde pendulet stabilt.