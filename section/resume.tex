Dette projekt undersøger og udarbejder et køretøj som kan balancere en stang. Denne stang balanceres ved hjælp af en sensor som delvist sidder på stangen og på køretøjet. Sensoren består af 3 spoler i alt, hvor den ene fungerer som den inducerende, og de to andre som sensorer. Køretøjet prøver, baseret på dette, at balancere stangen ved at modvirke stangens bevægelse. Dvs. når stangen vælter mod venstre, vil køretøjet rette den op ved selv at køre mod venstre. Det er vigtigt at køretøjet er forholdsvis hurtigt, da den ellers ikke ville kunne rette stangen op før det er for sent. Derfor er der påmonteret en motor med relativ stor kraft, sammen med et gearingsforhold som gør den både hurtig, men samtidig stærk. \\

For at undgå yderligere modstand, er dette køretøj batteridrevet. Dette gør den en del mere mobil, da man ikke er afhængig af en strømforsyning. \\

For at dette system fungerer, er det nødt til at undeholde en regulator. Systemet i denne rapport indeholder en PD-regulator, da denne var den mest passende. Dette får køretøjet til at være meget mere stabilt, end hvis den f.eks. kun havde haft en P-regulator. \\

\husk{fraannk}{Skal måske fyldes lidt mere på?}