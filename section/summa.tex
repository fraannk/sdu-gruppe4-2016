\section{Instrumenteringsforstærker}\label{sec:summa}
For at få en retning ud af spolesignalerne, anvendes instrumenteringsforstærkeren AD623. Denne IC har to input pins - en til hver af modtagerspolerne.
Netop instrumenteringsforstærkeren AD623, er valgt frem andre operationsforstærker typer, af grundlæggende årsager; for det første er det en differensforstærker, hvilket betyder at den forstærker differensspændingen som fås ved $v_{udgangssignal} = v_{in_1} - v_{in_2}$. Da instrumenteringsforstærkeren internt består af flere operationsforstærkere har den en høj indgangsimpedans, og samtadig kan forstærkningen styres med blot en gainmodstand, hvilket sparer plads på print boardet.

\begin{figure}[h!]
	\centering
	\includegraphics[width=1\textwidth]{billeder/instr_png.png}
	\caption{Billedet her viser outputtet fra instrumenteringsforstærkeren, hvor spolerne er i en tilfældig position.}
	\label{fig:filter_out}
\end{figure}

\subsection{Design}
Det samlede kredsløb for instrumenteringsforstærkeren består af en AD623, en gain modstand, samt et potentiometer. Dertil er der påsat to afkoblingskondensatorer. Kredsen forsynes med $\pm 7 \si{\volt}$, da det er max output fra batterierne. 

\husk{Kenneth}{Billede af instrumenterings kredsløb} 

\subsection{Beregninger}
\husk{Kenneth}{Find ud af spændingen før forstærkning. Også i filteret}
Da indgangssignalerne til instrumenteringsforstærkeren er meget lave, anvendes en gainmodstand for at forstærke signalet.

Der tilstræbes en forstærkning på 4 - 5 gange. Forstærkningen udregnes med ligning \ref{eq:GainModstand}.
\husk{Kenneth}{Link til datablad}
\begin{align}
	R_G & = \frac{100 \si{\kilo\ohm}}{G-1} \label{eq:GainModstand}
\end{align}
hvor en gainmodstand på 22 \si{\kilo\ohm} giver en forstærkning på 4.5.