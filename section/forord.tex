\chapter{Indledning}

\section{Forord}\label{sec:forord}
Udarbejdningen af dette projekt er blevet udført af 6 studerende på 3. semester ved Syddansk Universitet, Teknisk Fakultet fra uddannelserne elektronik og datateknik samt elektrisk energiteknologi.
Projektet er en sammenfatning af de opnåede fagligheder fra semesterets undervisning: elektronik, elektromagnetisme, regulering, kredsløbsteknik og matematik.

Projektet går ud på at balancere en stang i et én dimensionelt rum ved hjælp af en bil. Under projektforløbet har gruppen valgt, at uddeligere arbejdesopgaverne. Simon og Søren tog sig af det elektromagnetiske, hvor de fik designet spolerne og spoleholderne. Kenneth og Nikolaj fik lavet kredsløbene der gjorde det muligt, at få et brugbart signal til motorstyringen. Jörn analyserede det fysiske system; pendulets bevægelse i forhold til bilen, derudover motorstyringen og dimensioneringen af regulatoren.

Under hele projektforløbet har gruppen fulgt med i den enkeltes arbejdesopgaver, dels for at hjælpe hinanden og for at få en bedre samlet forståelse af projektet, og ikke mindst for at holde overblik i forhold til tidsplanen. 
\\ \\
Rapporten er bestående af samlet XX sider, hvoraf XX er tomme sider grundet layoutmæssige valg. 