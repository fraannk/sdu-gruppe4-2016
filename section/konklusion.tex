\chapter{Konklusion} \label{kap:konklusion}
En klassisk modelfremstilling af det fysiske system kunne give den nødvendige model til at beskrive det inverterede pendul.
Det viste sig muligt, at ved måling og beregninger, at bestemme dynamikken for vogn og motor.
Sammen med overførelsesfunktionerne fra de elektriske kredse i systemet, var det muligt at danne et samlet reguleringssystem og ved klassisk reguleringsteori at fremstille og dimensionere en P-lead regulator, der kunne stabilisere systemet.
Det blev vist at pendulets vinkel kan bestemmes, ved at anvende elektromagnetiske felter frembragt af spoler, hvor der blev opstillet en tilnærmet teoretisk model, der kunne eftervises.
Det blev også eftervist, hvordan magnetiske felter kobler sig imellem 2 spoler.
I det elektriske kredsløb viste en signalbehandling i form af et filter, en ensretning og signalsammenligning sig tilstrækkeligt for at kunne bruge transducerens signal til regulering.
Den valgte motorstyring blev designet så vognen kunne bevæge sig efter hensigten.
Den nødvendige forsyning, til de enkelte kredsløb, kunne med en spændingsregulator og en shunt-regulator realiseres.
Alle nødvendige konstruktionsdele til vognen, kunne med fordel fremstilles ved CAD og 3D-printer teknologi.

Det kan derfor konkluderes at alle punkterne i problemstillingen er blevet besvaret, at det udarbejdede produkt er funktionelt samt overholder de stillede krav for projektet. 