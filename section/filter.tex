\section{Båndpassfilter}\label{sec:filter}
Til at fjerne støj og andre udefrakommende forstyrrelser i systemet, skal der laves et filter.
Der vil her blive designet et aktivt båndpas filter, da der kun ønskes at én bestemt frekvens forstærkes.
Fordelen ved det aktive filter er at den kan både forstærke og filtrerer.

\subsection{Design}
Den valgte opstilling der anvendes er taget fra figur 5.1-4 \cite[side. 208]{Huelsman1993}.
\husk{Nikolaj}{Indsæt schematic af det aktive filter fra bogen}
Filteret er designet så modstanden $R_5$ gør forstærkningen ved centerfrekvens til en fri variable.
Heraf designes filteret efter en ønsket båndbredde, centerfrekvens og centerfrekvens forstærkning. Q er systemets godhed og anvendes til at bestemme bredden på båndbredden. 
$\omega_c$ er centerfrekvensen der bestemmer ved hvilken frekvens filterets maksimale forstærkning optræde.
$H_o$ er forstærkningen ved den ønskede centerfrekvens.

\subsection{Beregninger}
Der ønskes her en centerfrekvens forstærkning på 3 og en godhed på 3. 
Centerfrekvensen der skal have den største forstærkning er frekvensen der kommer på udgangen af timeren.
Heraf skal der vælges kondensator værdier, da der ikke kan regnes modstandsværdier uden. 
Det er her besluttet at begge kondensatorer skal være lige store for at simplificerer udregningerne.
\begin{align}
	C_{10} & = 470 \si{\pico\farad} \nonumber \\
	H_o & = 3 \nonumber \\
	Q & = 3 \nonumber \\
	F_c & = 46.936 \si{\kilo\hertz} \nonumber \\
	\omega_c & = 2 \cdot \pi \cdot F_c
\end{align}

Heraf udregnes båndbredden for filteret og filtermodstandene. \cite[Side. 209]{Huelsman1993}
\begin{align}
	B & = \frac{F_c}{Q} \\
	R_8 & = \frac{Q}{\omega_n \cdot C_{filter} \cdot H_o } \\
	R_9 & = \frac{Q}{ \omega_n \cdot C_{filter} \cdot \left( 2 \cdot Q^2 - H_o \right) } \\
	R_{10} & = \frac{2 \cdot Q}{ \omega_n \cdot C} \\
	B & = 15.645 \si{\kilo\hertz} \nonumber \\
	R_8 & = 7.215 \si{\kilo\hertz} \nonumber \\
	R_9 & = 1.430 \si{\kilo\ohm} \nonumber \\
	R_{10} & = 43.288 \si{\kilo\ohm} \nonumber 
\end{align}
De teoretiske modstandsværdier skal tilpasses til de SMD modstande der er til rådighed.
\begin{align}
	R_8 & = 6.8 \si{\kilo\ohm} \nonumber \\
	R_9 & = 1.5 \si{\kilo\ohm} \nonumber \\
	R_{10} & = 47 \si{\kilo\ohm} \nonumber
\end{align}
Disse ændringer bruges til at regne ud hvordan dette påvirker det praktiske system. Til udregning af dette omskrives ligningerne 13a, 13b og 13c. \cite[Side. 208]{Huelsman1993}
\begin{align}
	F_c & = \frac{1}{2 \cdot \pi \cdot C} \cdot \sqrt{\frac{R_8+R_9}{R_8 \cdot R_9 \cdot R_{10}}} \\
	H_o & = \frac{1}{\left( \frac{R_1}{R_3} \right) \cdot 2} \\
	B & = \frac{F_c}{Q} \\
	F_c & = 44.557 \si{\kilo\ohm} \nonumber \\
	H_o & = 3.5 \nonumber \\
	B & = 14.9 \si{\kilo\ohm} \nonumber
\end{align}
En forøgelse af forstærkningen er acceptabel så længe instrumenteringsforstærkeren ikke går i mætning, hvilket den er designet til ikke at gøre. En mindre centerfrekvens og båndbredde findes også acceptabel. Båndbredden er stor nok til at systemet ikke bliver bemærkelsesværdigt langsomt, og centerfrekvensen varrierer ikke så meget fra den ønskede at den bliver dæmpet.
