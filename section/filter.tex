\section{Båndpassfilter}\label{sec:filter}
Til at fjerne støj og andre udefrakommende forstyrrelser i systemet, skal der laves et filter.
Der vil her blive designet et aktivt båndpas filter, da der kun ønskes at én bestemt frekvens forstærkes.
Fordelen ved det aktive filter er at den kan både forstærke og filtrere.

\subsection{Design}
\begin{wrapfigure}{r}{0.5\textwidth}
	\centering
	\includegraphics[width=0.48\textwidth]{billeder/ActiveFilter.png}
	\caption{Diagram over det aktive båndpasfilter.}
	\label{fig:ActiveFilter}
\end{wrapfigure}
Opstillingen af filteret ses på figur \ref{fig:ActiveFilter}.
Filteret er designet så modstanden $R_5$ gør forstærkningen ved resonansfrekvens til en fri variable.
Heraf designes filteret efter en ønsket båndbredde, resonansfrekvens og resonansfrekvens forstærkning. Q er systemets godhed og anvendes til at bestemme bredden på båndbredden. 
$f_0$ er resonansfrekvensen der bestemmer ved hvilken frekvens filterets maksimale forstærkning optræder.
$H_o$ er forstærkningen ved den ønskede centerfrekvens.
\subsection{Beregninger}
For at udregne modstandsværdier skal der vælges størrelse på kondensatorene, forstærkning ved resonansfrekvens og filterets godhed.
Det er her besluttet at begge kondensatorer skal være lige store for at simplificerer udregningerne.
Kondensatoren vælges til $C_{10} = 470 \si{\pico\farad}$, godheden vælges til $Q_g = 3$, resonansfrekvens forstærkingen vælges til $H_o = 3$ og resonansfrekvensen vælges til samme frekvens som på udgangen af timeren, $f_0 = \SI{46.936}{\kilo\hertz}$.
Heraf udregnes båndbredden for filteret og filtermodstandene \cite[Side. 209]{Huelsman1993}.
Til udregning af resonansvinkelfrekvensen henvises til ligning \ref{eq:vinkelfrekvens}.
\begin{align}
	B_w & = \frac{f_0}{Q_g}
	\end{align}
Filterets båndbredde er givet ved forholdet mellem resonansfrekvensen og filterets godhed.
\begin{align}
	R_8 & = \frac{Q_g}{\omega_c \cdot C_{10} \cdot H_o } \label{eq:R8}
	\end{align}
Filterets indgangsmodstand er med til at påvirke systemets centerfrekvens forstærkning, men den kan ikke alene gøre den til en fri variable. 
\begin{align}
	R_9 & = \frac{Q_g}{ \omega_c \cdot C_{10} \cdot \left( 2 \cdot Q_g^2 - H_o \right) } \label{eq:R9}
	\end{align}
Modstanden $R_9$ er med til at gøre forstærkningen til en fri variabel, hvis denne ikke var med, så havde filterets forstærkning været afhængig af godheden.
\begin{align}
	R_{10} & = \frac{2 \cdot Q_g}{ \omega_c \cdot C} \label{eq:R10}
\end{align}
Feedback modstanden regnes ved ligning \ref{eq:R10}.
Dette medfører. $R_8 = \SI{7.215}{\kilo\ohm}$, $R_9 = \SI{1.430}{\kilo\ohm}$, $R_{10} = \SI{43.288}{\kilo\ohm}$ og $B_w = \SI{15.645}{\kilo\hertz}$.
Da disse modstands værdier ikke kan realiseres i SMD, vælges derfor tilnærmede størrelser.
De tilnærmede komponentværdier kommer til at være. $R_8 = \SI{6.8}{\kilo\ohm}$, $R_9 = \SI{1.5}{\kilo\ohm}$ og $R_{10} = 47 \si{\kilo\ohm}$.
De tilpassede komponenter medfører at systemet har en anden karakteristik end den udregnede. 
Til udregning anvendes ligningerne 13a, 13b og 13c \cite[Side. 208]{Huelsman1993}. 
\begin{align}
	f_0 & = \frac{\sqrt{1+\frac{R_9}{R_8}}}{\left( 2 \cdot \pi \right) \cdot \sqrt{R_9 \cdot R_{10} \cdot C_{10}^2}}
	\end{align}
Centerfrekvensen ændrer sig hvis de teoretiske værdier ikke kan opfyldes. Denne ligning bruges til at udregne hvor den nye centerfrekvens kommer til at ligge.
\begin{align}
	Q_g & = \left( \frac{2 \cdot \sqrt{\frac{R_9 \cdot C_{10}}{R_{10} \cdot C_{10}}}}{\sqrt{1+\frac{R_9}{R_8}}} \right)^{-1}
	\end{align}
Når kondensatorene er lige store så har de ingen indflydelse på filterets godhed. Denne ligning er kun afhængig af modstandsværdierne i dette tilfælde.
\begin{align}
	H_o & = \frac{\frac{R_{10}}{R_8}}{1+\frac{C_{10}}{C_{10}}}
\end{align}
Tilsvarende for godheden så har kondensatorene ikke nogen påvirkning når de er lige store.
Ændringen er her udelukkende afhængig af indgangs og feedback modstand.
Dette medfører at systemets karakteristik er. 
$H_o = \num{3.5}$, $f_0 = \SI{44.557}{\kilo\hertz}$, $Q_g = \num{3.1}$ og $B_w = \SI{14.4}{\kilo\hertz}$. 
En forøgelse på $16 \%$ på forstærkningen kommer ikke til at blive et problem så længe instrumenterings forstærkeren ikke går i mætning. 
Centerfrekvens ændringen er acceptabel, da den ikke ligger så langt fra den teoretiske at det forårsager den dæmpning af spændingen.
En højere godhed medfører at båndbredden bliver mindre, men da båndbredden stadig er stor, så kommer det ikke til at gøre systemet signifikant langsommere.
\begin{figure}[h!]
	\centering
	\includegraphics[width=1\textwidth]{billeder/filter_in_png.png}
	\caption{Her ses signalet fra modtagerspolen før filteret.}
	\label{fig:filter_in}
\end{figure}
\begin{figure}[h!]
	\centering
	\includegraphics[width=1\textwidth]{billeder/filter_out_png.png}
	\caption{Her ses signalet efter filteret, hvor signalet er forstærket i filteret.}
	\label{fig:filter_out}
\end{figure}