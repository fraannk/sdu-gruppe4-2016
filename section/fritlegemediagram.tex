\section{Fritlegemediagram af pendul}\label{sec:sec_fritlegemediagram}

Opsummering af kræfter på vognen som ses i figur xx
\begin{alignat}{3}
&\hat{i} : \quad && F_c + T sin{\theta} && = m_v \ddot{x} \label{eq:vogn_x}\\
&\hat{j} : \quad && -g m_v + N && = 0 
\end{alignat}

Opsummering af kræfter på pendulet som ses i figur xx
\begin{alignat}{3}
&\hat{i} : \quad &&-T sin{\theta} &&= m_p a_{px}\label{eq:apx}\\
&\hat{j} : \quad &&-T cos{\theta} - m_p g &&= m_p a_{py}\label{eq:apy} 
\end{alignat}

Relativ acceleration af pendul i polær koordinater angives som 
%\husk{JJ}{Tegning med retningsvektorer angivet for pendulet i polær-koordinator}
\begin{align}
\vec{a}_p &= \vec{a}_v + \vec{a}_{p/v} \nonumber \\
&= \ddot{x} \hat{i} + \left( L\ddot{\theta}\hat{\phi} - L\dot{\theta}^2\hat{r} \right)
\end{align}

Accelerations vektoren $\vec{a}_p$, for pendulet omskrives nu fra polær-koordinator til kartesisk-koordinator  
\begin{align}
\vec{a}_p &=  \ddot{x} \hat{i} 
				+ L\ddot{\theta} \left( \cos{\theta}\hat{i} - \sin{\theta}\hat{j} \right) 
				- L\dot{\theta}^2 \left( \sin{\theta}\hat{i} + \cos{\theta}\hat{j} \right) \label{eq:ap}
\end{align} 

Accelerationen for pendulet, $\vec{a}_{px}$ og $\vec{a}_{py}$, i ligning \ref{eq:apx} og \ref{eq:apy} kan nu erstattes, med den relative acceleration fundet i ligning \ref{eq:ap}
\begin{align}
-T\sin{\theta} &= m_p \left( \ddot{x} + L\ddot{\theta}\cos{\theta} - L\dot{\theta}^2\sin{\theta} \right)  \label{eq:pendul_x2}\\
-T\cos{\theta} - m_p g &=  -m_p \left( L\ddot{\theta}\sin{\theta} + L\dot{\theta}^2\cos{\theta}  \right) \label{eq:pendul_y2}
\end{align} 

Trækkræfter $T$ fjernes fra pendul ligningerne \ref{eq:pendul_x2} og \ref{eq:pendul_y2} ved gensidig at gange med $-\cos{\theta}$ og $\sin{\theta}$. 
\begin{align}
T\sin{\theta}\cos{\theta} &=   -m_p \cos{\theta} \left( \ddot{x} + L\ddot{\theta}\cos{\theta} - L\dot{\theta}^2\sin{\theta} \right) \label{eq:pendul_x3} \\
-T\cos{\theta}\sin{\theta} - m_p g &=  -m_p \sin{\theta} \left( L\ddot{\theta}\sin{\theta} + L\dot{\theta}^2\cos{\theta}\right) \label{eq:pendul_y3}
\end{align}

Ligning \ref{eq:pendul_x3} og \ref{eq:pendul_y3} adderes
\begin{align}
-m_p g \sin{\theta}    &= - m_p \ddot{x}\cos{\theta}
						- m_p L\ddot{\theta}\cos^2\theta\
						\hcancel[red]{+ m_p L \dot{\theta}^2\cos{\theta}\sin{\theta}} \nonumber\\
					   &\quad - m_p L\ddot{\theta}\sin^2\theta\
					    \hcancel[red]{- m_p L \dot{\theta}^2\cos{\theta}\sin{\theta}} \nonumber\\
					   &=- m_p \cos{\theta}\ddot{x} - m_p L \ddot{\theta} \left(\sin^2 \theta \cos^2 \theta \right) \nonumber \\
					   &=- m_p \cos{\theta}\ddot{x} - m_p L \ddot{\theta} \label{eq:pendul_mellem}
\end{align}

I ligning \ref{eq:vogn_x} fjernes $T$ ved at substituere med ligning \ref{eq:pendul_x2}
\begin{align}
F_c - m_p \ddot{x} - m_p L\ddot{\theta}\cos{\theta} + m_p L\dot{\theta}^2\sin{\theta} &= m_v \ddot{x} \Leftrightarrow \nonumber \\
F_c - m_p L\ddot{\theta}\cos{\theta} + m_p L\dot{\theta}^2\sin{\theta} &= (m_v + m_p)  \ddot{x} \label{eq:vogn_mellem}
\end{align}


Ligningerne af det fysiske system for pendul og vogn fremstår i nu i \ref{eq:pendul_mellem} og \ref{eq:vogn_mellem}. Disse to ligninger er ikke lineære. En forudsætning for at kunne gennemføre en linearisering er at antage, at udefrakommende påvirkninger til systemet er så små, at små forstyrrelser kun give anledning til små ændringer af vinkelen $\theta$ og en approksimmering kan antages som
\begin{align}
\sin{\theta} \approxeq \theta \quad og \cos{\theta} \approxeq 1
\end{align} 

Ligeledes kan udtrykket $\dot{\theta}^2$ reduceres til
\begin{align}
\dot{\theta}^2 = \left( \dfrac{d}{dt}\theta \right)^2 =  \left( \dfrac{d}{dt}\sin{\theta} \right)^2 = \left(\cos{\theta}\right)^2 \approxeq (1)^ = 1
\end{align}


Det ønskes at få et udtryk der forbinder position $x$ med vinklen $\theta$ for pendulet, så ligning \ref{eq:pendul_mellem} forkortes. Pendulets masse $m_p$ forkortes også bort. 
\begin{align}
-m_pg\theta &= -m_p\ddot{x}-m_p L \ddot{\theta} \Leftrightarrow\\
\ddot{x} &= g \theta -L \ddot{\theta} \label{eq:pendul_lin}
\end{align}

Nu tages \ref{eq:pendul_lin} og indsættes i ligning \ref{eq:vogn_mellem}
\begin{align}
F_c - m_p L\ddot{\theta} + m_p L\theta &= (m_v + m_p)(g \theta -L \ddot{\theta}) \nonumber \\
F_c  - m_p L\ddot{\theta} + m_p L\theta &= m_v g \theta +  m_p g \theta - m_v L \ddot{\theta} - m_p L \ddot{\theta}
\end{align}

I ovenstående udtryk, erstattes $F_c = ma_c$ for og reduceres
\begin{align}
m a_c - \hcancel[red]{ m_p L\ddot{\theta}} + m_p L\theta &= (m_v + m_p) g \theta - m_v L \ddot{\theta} - \hcancel[red]{ m_p L\ddot{\theta}} \Leftrightarrow \nonumber \\
m_v L \ddot{\theta} - m g \theta &= - m_p L\theta -m a_c \label{eq:system_final}
\end{align} 

Ligning \ref{eq:system_final} er således den endelige beskrivelse af pendul og vogn hvori systemets overførelses funktion kan findes. På højre side ses de eksterne forstyrrelser. $m_pL\theta $ er påvirkningen af en lille ændring af pendulet, og $m a_c$ er en ændring i vognens acceleration. Ved at Laplace transformere ligningen, findes $\theta(s)$
\begin{align}
m_vLs^2\theta(s) - mg = -m_pL\theta(s) -mA_c(s) \\
\theta(s)\left(m_vLs^2 - mg \right) = -m_pL\theta(s) -mA_c(s) \\
\theta(s) = \frac{1}{m_vLs^2 - mg }\left[-m_pL\theta(s) -mA_c(s)\right]
\end{align} 
